\documentclass[10pt,a4paper]{cv}
\geometry{left=1cm,right=11.2cm,marginparwidth=9cm,marginparsep=1.2cm,top=1cm,bottom=1cm}

\usepackage[utf8]{inputenc}
\usepackage[T1]{fontenc}
\usepackage[default]{lato}
\usepackage{hyperref}

\definecolor{VividPurple}{HTML}{0085c3}
\definecolor{SlateGrey}{HTML}{2E2E2E}
\definecolor{LightGrey}{HTML}{666666}
\colorlet{heading}{VividPurple}
\colorlet{accent}{VividPurple}
\colorlet{emphasis}{SlateGrey}
\colorlet{body}{LightGrey}

\usepackage{xcolor}
\hypersetup{
    colorlinks,
    linkcolor={VividPurple},
    citecolor={VividPurple},
    urlcolor={VividPurple}
}


% itemize and rating marker for \cvskill 
\renewcommand{\itemmarker}{{\small\textbullet}}
\renewcommand{\ratingmarker}{\faCircle}


%\usepackage{hyperref}

\begin{document}
%
\name{Nikolai Gaiduchenko}
\tagline{Curriculum Vitae}
\photo{3.5cm}{main}
%
\personalinfo{%
  \begin{minipage}{4cm}
  	\email{\href{mailto:Gaiduchenko.NE@gmail.com}{\textcolor{SlateGrey}{Gaiduchenko.NE@gmail.com}}}
   
   \github{\href{https://github.com/Yozh2}{\textcolor{SlateGrey}{github.com/Yozh2}}}
   
   \end{minipage}%
   \hfill%
   \begin{minipage}{5.5cm}
   
   \phone{+7-925-450-82-99}
   \location{Moscow, Russia}
  \end{minipage}
}
	

\begin{adjustwidth}{}{-10cm}
\makecvheader
\end{adjustwidth}

\cvsection[page1sidebar]{work experience}

% MIL
\cvevent{\textbf{Data Scientist}}{\href{https://mipt.ru/science/labs/machine-intelligence/}{Machine Intelligence Laboratory by K.V. Vorontsov}}{09.18 -- 11.18}{MIPT, Dolgoprudny, Russia}
One-click Deep Learning pipeline for synthetic document dataset generation

\divider

% Samsung
\cvevent{\textbf{Engineer Assistant}}{\href{https://www.samsung.com/ru/aboutsamsung/careers/working-at-rnd/}{Samsung R\&D,~AI Algorithms Lab}}{06.18 -- 09.18}{Moscow, Russia}
Deep learning and computer vision for self-driving robot

\divider

% Работа на Алмазе
\cvevent{\textbf{Technician, 2nd category}}{\href{https://ras[letin.com}{SPA (NPO) <<Almaz>>, R\&D dept.}}{12.18 -- present}{Moscow, Russia}
Algorithm of extrapolation of motion parameters of hypersonic aircrafts using neural networks

\divider

% Лаба на Алмазе
\cvevent{\textbf{Laboratory Assistant}}{\href{http://miptdesigncenter.tilda.ws}{Laboratory of Design of Special Computer Systems Architectures}}{10.17 -- present}{MIPT, Dolgoprudny, Russia}
Sub-array antenna modeling app for CAD Radar Python package \textit{<<PyPhased>>}

\cvsection{Education and training}
\cvevent{}{\href{https://mipt.ru/english/}{Moscow Institute of Physics and Technology}}{2015 -- present $(4^{th} year)$}{Russia, Dolgoprudny}
Department of Radio Engineering and Cybernetics

BSc, Applied Mathematics and Physics,
GPA: 8.8/10.0

\begin{itemize}
\item Mathematics (Linear Algebra, Differential Equations, Theory of Probability, Mathematical Statistics)
\item General Physics, Theoretical Mechanics, Field Theory, Quantum Mechanics
\item Engineering, CAD, Micro Controllers, Radar Applications, Digital Signal Processing, Signal Theory
\item Cryptography, Information Security Theory, Information Theory
\item Computer Science, Computational Mathematics, ML algorithms and principles, Computer Vision
\end{itemize}

% Губернский лицей-интернат для одарённых детей
\cvevent{}{\href{http://licpnz.ru}{Regional Lyceum for Gifted Children}}{2013 -- 2015}{Russia, Penza}
\divider

% KSL
\cvevent{}{\href{https://academy.kaspersky.ru/subprojects/summerlab}{Kaspersky Cybersecurity Summer Lab}}{2017}{Russia, Kaluga State}{}

% ЛОШ
%\cvevent{}{\href{https://it-edu.mipt.ru/ru/школьникам/school-olymp/summer15}{MIPT Summer Olympiad School}}{2015}{Russia, Dolgoprudny}{}

% ЗФТШ
\cvevent{}{\href{http://www.school.mipt.ru}{MIPT Extramural Physical-Technical School}}{2011-2015}{Russia, Dolgoprudny}{}
 
% Нанограды   
%\cvevent{}{\href{http://schoolnano.ru/nanograd2017}{Nanograd Summer School (by RUSNANO School League)}}{2011-2014}{Russia: Penza, Kazan, Moscow, Samara}

\clearpage
\end{document}
