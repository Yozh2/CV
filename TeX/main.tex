\documentclass[10pt,a4paper]{cv}
\geometry{left=1cm,right=11.2cm,marginparwidth=9cm,marginparsep=1.2cm,top=1cm,bottom=1cm}

\usepackage[utf8]{inputenc}
\usepackage[T1]{fontenc}
\usepackage[default]{lato}
\usepackage{hyperref}

\definecolor{VividPurple}{HTML}{0085c3}
\definecolor{SlateGrey}{HTML}{2E2E2E}
\definecolor{LightGrey}{HTML}{666666}
\colorlet{heading}{VividPurple}
\colorlet{accent}{VividPurple}
\colorlet{emphasis}{SlateGrey}
\colorlet{body}{LightGrey}

\usepackage{xcolor}
\hypersetup{
    colorlinks,
    linkcolor={VividPurple},
    citecolor={VividPurple},
    urlcolor={VividPurple}
}


% itemize and rating marker
% for \cvskill 
\renewcommand{\itemmarker}{{\small\textbullet}}
\renewcommand{\ratingmarker}{\faCircle}


%\usepackage{hyperref}

\begin{document}
%
\name{Nikolai Gaiduchenko}
\tagline{Curriculum Vitae}
\photo{3.5cm}{main}
%
\personalinfo{%
  \begin{minipage}{4cm}
  	\email{\href{mailto:Gaiduchenko.NE@gmail.com}{\textcolor{SlateGrey}{Gaiduchenko.NE@gmail.com}}}
   
   \github{\href{https://github.com/Yozh2}{\textcolor{SlateGrey}{github.com/Yozh2}}}
   
   \end{minipage}%
   \hfill%
   \begin{minipage}{5.5cm}
   
   \phone{+7-925-450-82-99}
   \location{Russia, Moscow}
  \end{minipage}
 

	
}
	

\begin{adjustwidth}{}{-10cm}
\makecvheader
\end{adjustwidth}


\cvsection[page1sidebar]{Education and training}

\cvevent{}{\href{https://mipt.ru/english/}{Moscow Institute of Physics and Technology}}{2015 -- present $(3^{d} year)$}{Russia, Dolgoprudny}
Department of Radio Engineering and Cybernetics

BSc, Applied Mathematics and Physics,
GPA: 8.8/10.0

\begin{itemize}
\item Mathematics (Linear Algebra, Differential Equations, Theory of Probability, Mathematical Statistics)
\item General Physics, Theoretical Mechanics, Field Theory, Quantum Mechanics.
\item Engineering, CAD, Micro Controllers, Radar Applications, Digital Signal Processing, Signal Theory.

\item Computer Science, Computational Mathematics, ML algorithms and principles.

\end{itemize}
\cvevent{}{\href{http://licpnz.ru}{Regional Lyceum for Gifted Children}}{2013 -- 2015}{Russia, Penza}
\divider

% KSL
\cvevent{}{\href{https://academy.kaspersky.ru/subprojects/summerlab}{Kaspersky Cybersecurity Summer Lab}}{2017}{Russia, Kaluga State}{}

% ЛОШ
\cvevent{}{\href{https://it-edu.mipt.ru/ru/школьникам/school-olymp/summer15}{MIPT Summer Olympiad School}}{2015}{Russia, Dolgoprudny}{}
%ЗФТШ
\cvevent{}{\href{http://www.school.mipt.ru}{MIPT Extramural Physical-Technical School
}}{2011-2015}{Russia, Dolgoprudny}{}

\cvevent{}{\href{http://schoolnano.ru/nanograd2017}{Nanograd Summer School (by RUSNANO School League)}}{2011-2014}{Russia: Penza, Kazan, Moscow, Samara}


\cvsection{Relevant courses}

\cvsubsection{Advanced C programming in UNIX environment}
by D. V. Lunev, core QEMU, Parallels, OpenVZ developer

\cvsubsection{Machine Learning and Data Analysis}
on Courcera by MIPT \& Yandex

\cvsubsection{Building Deep Neural Networks in Python}
by Andrey Sozykin

\cvsection{Work Experience}
\cvsubsection{MIPT Laboratory Assistant}
Laboratory of Design of Special Computer Systems Architectures
\begin{itemize}
\item Sub-array modeling of a radar antenna app (Stack: Python3, PyQt5, scipy, numpy)
\end{itemize}

\cvsubsection{Electrical engineer}
AVR, STM32 microcontrollers driven devices

\cvsubsection{Python Developer}
Web and deploy tools for MIPT CS Department, lectures on cyber-security, Git systems.
\clearpage
\end{document}
