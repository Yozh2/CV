\documentclass[10pt,a4paper]{cv}
\geometry{left=1cm,right=11.2cm,marginparwidth=9cm,marginparsep=1.2cm,top=1cm,bottom=1cm}

%\usepackage[T2A, T1]{fontenc}


\usepackage[utf8]{inputenc}
\usepackage[russian]{babel}
\usepackage[default]{lato}
\usepackage{hyperref}
\fontfamily{sans-serif}

\definecolor{VividPurple}{HTML}{0085c3}
\definecolor{SlateGrey}{HTML}{2E2E2E}
\definecolor{LightGrey}{HTML}{666666}
\colorlet{heading}{VividPurple}
\colorlet{accent}{VividPurple}
\colorlet{emphasis}{SlateGrey}
\colorlet{body}{LightGrey}

\usepackage{xcolor}


\hypersetup{
    colorlinks,
    linkcolor={VividPurple},
    citecolor={VividPurple},
    urlcolor={VividPurple}
}


% itemize and rating marker for \cvskill 
\renewcommand{\itemmarker}{{\small\textbullet}}
\renewcommand{\ratingmarker}{\faCircle}


%\usepackage{hyperref}

\begin{document}
%
\name{Николай Гайдученко}
\tagline{Резюме}
\photo{3.5cm}{main}
%
\personalinfo{%
  \begin{minipage}{4cm}
  	\email{\href{mailto:Gaiduchenko.NE@gmail.com}{\textcolor{SlateGrey}{Gaiduchenko.NE@gmail.com}}}
   
   \github{\href{https://github.com/Yozh2}{\textcolor{SlateGrey}{github.com/Yozh2}}}
   
   \end{minipage}%
   \hfill%
   \begin{minipage}{5.5cm}
   
   \phone{+7-925-450-82-99}
   \location{Москва, Россия}
  \end{minipage}
}
	

\begin{adjustwidth}{}{-10cm}
\makecvheader
\end{adjustwidth}

\cvsection[page1sidebar_ru]{ОПЫТ РАБОТЫ}

% MIL
\cvevent{\textbf{Data Scientist}}{\href{https://mipt.ru/science/labs/machine-intelligence/}{Лаборатория Машинного Интеллекта К.В.  Воронцова}}{09.18 -- 11.18}{МФТИ, Долгопрудный, Россия}
Алгоритм генерации синтетических датасетов с документами на основе Глубокого Обучения

\divider

% Samsung
\cvevent{\textbf{Engineer Assistant}}{\href{https://www.samsung.com/ru/aboutsamsung/careers/working-at-rnd/}{Samsung R\&D,~AI Algorithms Lab}}{06.18 -- 09.18}{Москва, Россия}
Глубокое Обучение и машинное зрение для самоуправляемого робота

\divider

% Работа на Алмазе
\cvevent{\textbf{Техник 2й Категории}}{\href{https://ras[letin.com}{НПО <<Алмаз>>, ОКБ 1}}{12.18 -- по н. вр.}{Москва, Россия}
Алгоритм экстраполяции параметров движения гиперзвуковых летательных аппаратов с применением нейронных сетей

\divider

% Лаба на Алмазе
\cvevent{\textbf{Лаборант}}{\href{http://miptdesigncenter.tilda.ws}{Лаборатория моделирования и проектирования архитектур специальных вычислительных систем}}{10.17 -- по н. вр.}{МФТИ, Долгопрудный, Россия}
Программа для моделирования АФАР для САПР РЛС Python-пакета  \textit{<<PyPhased>>}

\cvsection{ОБРАЗОВАНИЕ}
\cvevent{}{\href{https://mipt.ru/english/}{Московский Физико-Технический Институт}}{2015 -- по н. вр. $(4^{й} курс)$}{Долгопрудный, Россия}
Факультет Радиотехники и Кибернетики
Бакалавриат, <<Прикладные Математика и Физика>>,
ср. балл: 8.8/10.0

\begin{itemize}
\item Математика (линейная алгебра, дифференциальные уравнения, теория вероятностей, мат. статистика)
\item Общая физика, теоретическая механика, теория поля, квантовая механика
\item Основы инженерного дела, САПР, микроконтроллеры, основы радиолокации, цифровая обработка сигналов, теория сигналов
\item Криптография, защита информации, теория информации
\item Программирование, вычислительная математика, основы и алгоритмы машинного обучения, компьютерное зрение
\end{itemize}

% Губернский лицей-интернат для одарённых детей
%\cvevent{}{\href{http://licpnz.ru}{Губернский лицей-интернат для одарённых детей}}{2013 -- 2015}{Пенза, Россия}
%\divider

% KSL
%\cvevent{}{\href{https://academy.kaspersky.ru/subprojects/summerlab}{Kaspersky Cybersecurity Summer Lab}}{2017}{Калужская область, Россия}{}

% ЛОШ
%\cvevent{}{\href{https://it-edu.mipt.ru/ru/школьникам/school-olymp/summer15}{MIPT Summer Olympiad School}}{2015}{Russia, Dolgoprudny}{}

% ЗФТШ
%\cvevent{}{\href{http://www.school.mipt.ru}{Заочная Физико-Техническая Школа МФТИ}}{2011-2015}{Долгопрудный, Россия}{}
 
% Нанограды   
%\cvevent{}{\href{http://schoolnano.ru/nanograd2017}{Nanograd Summer School (by RUSNANO School League)}}{2011-2014}{Russia: Penza, Kazan, Moscow, Samara}

\clearpage
\end{document}
